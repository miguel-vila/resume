%%%%%%%%%%%%%%%%%%%%%%%%%%%%%%%%%%%%%%%%%
% Medium Length Professional CV
% LaTeX Template
%
% This template has been downloaded from:
% http://www.LaTeXTemplates.com
%
% Original author:
% Trey Hunner (http://www.treyhunner.com/)
%
% Important note:
% This template requires the resume.cls file to be in the same directory as the
% .tex file. The resume.cls file provides the resume style used for structuring the
% document.
%
%%%%%%%%%%%%%%%%%%%%%%%%%%%%%%%%%%%%%%%%%

%----------------------------------------------------------------------------------------
%	PACKAGES AND OTHER DOCUMENT CONFIGURATIONS
%----------------------------------------------------------------------------------------

\documentclass{resume} % Use the custom resume.cls style

\usepackage[left=0.75in,top=0.6in,right=0.75in,bottom=0.6in]{geometry} % Document margins
\usepackage{hyperref}
\name{Miguel Vil\'a Gonz\'alez} % Your name
\address{
	Fijo: (+571)~$\cdot$~540~$\cdot$~8404 \\ 
	M\'ovil: 310$\cdot$751$\cdot$4361 \\
    \href{https://github.com/miguel-vila/}{miguel-vila @ github} \\
	\url{miguelvilag@gmail.com}} % Your phone number and email

\def\uniandes{Universidad de Los Andes }
\def\dept{Dept. de Ingenier\'ia de Sistemas }
\def\depart_uniandes{\dept y Computaci\'on, \uniandes}

\begin{document}

\begin{rSection}{Perfil profesional}
Me considero una persona responsable, honesta y trabajadora. Procuro balancear soluciones eficientes y solidas con objetivos y necesidades concretas. Tengo habilidades para el trabajo en grupo y poseo iniciativa y curiosidad. Domino diversas tecnolog\'ias y soy capaz de aprender r\'apidamente otras nuevas.

\end{rSection}
%----------------------------------------------------------------------------------------
%	EDUCATION SECTION
%----------------------------------------------------------------------------------------

\begin{rSection}{Educaci\'on}

{\bf \uniandes} \hfill Enero de 2009 - Diciembre de 2012 \\ 
Ingenier\'ia de Sistemas y Computaci\'on \\

\end{rSection}

%\begin{rSection}{Intereses acad\'emicos}
%\begin{rSubsection}{}{}{}{}
%\item \textbf{Almacenamiento y procesamiento de grandes vol\'umenes de datos no estructurados:} Retos involucrados en el dise\~{n}o e implementaci\'on de sistemas que soporten el manejo de altas cantidades de informaci\'on no estructurada.
%\item \textbf{Procesamiento de lenguaje natural y miner\'ia de textos:} Aplicaciones de procesamiento de lenguaje natural que permitan generar valor a partir del an\'alisis de informaci\'on textual.
%\end{rSubsection}
%\end{rSection}
%----------------------------------------------------------------------------------------
%	WORK EXPERIENCE SECTION
%----------------------------------------------------------------------------------------

\begin{rSection}{Experiencia profesional}

\begin{rSubsection}{Seven4n}{Enero de 2013 - Presente}{Ingeniero de Software}{Bogot\'a, Colombia}
\item Implementaci\'on de un proceso de negocio, desarrollo de microflujos de BPEL y de mediaciones de servicios web.
\item Desarrollo de un sistema que orquestaba el consumo de diversos servicios web usando Scala y la librer\'ia de sistemas distribuidos y concurrentes Akka.
\item Desarrollo de un API REST y una SPA usando los frameworks Play (con Scala) y Ember.js. 
\item Desarrollo de un aplicativo para el sector de seguros usando Scala y Angular.js.
\end{rSubsection}

%------------------------------------------------

\begin{rSubsection}{\uniandes}{Agosto de 2011 - Mayo de 2012}{Monitor}{Bogot\'a, Colombia}
\item Monitor de la materia \textit{Dise\~{n}o y an\'alisis de algoritmos}.\\
Calificaci\'on de trabajos y ex\'amenes. Tutor\'ias previas a evaluaciones.
\end{rSubsection}

%------------------------------------------------

\begin{rSubsection}{CIACUA, \uniandes}{Marzo de 2011 - Agosto de 2011}{Ingeniero de Software}{Bogot\'a, Colombia}
\item Proceso de reingenier\'ia de software para el dise\~{n}o y simulaci\'on de redes de distribuci\'on de agua. Diagramaci\'on en UML y correcci\'on de bugs encontrados en cambios de versiones.
\end{rSubsection}

\end{rSection}

\begin{rSection}{Logros y reconocimientos}
\begin{rSubsection}{}{}{}{}
\item Primer puesto en los ex\'amenes de admisi\'on a la carrera de Ingenier\'ia de Sistemas en la Universidad Nacional de Colombia (Sede Bogot\'a, Octubre de 2008).
\item Beca \textsl{Quiero estudiar} para estudios de pregrado en Ingenier\'ia de Sistemas y Computaci\'on en la Universidad de Los Andes (Octubre de 2008).
\item Clasificado a la marat\'on regional latinoamericana de programaci\'on (ICPC), 2011.
\item Clasificado a la marat\'on regional latinoamericana de programaci\'on (ICPC), 2012.
\end{rSubsection}
\end{rSection}

\begin{rSection}{Educaci\'on complementaria}
\begin{rSubsection}{}{}{}{}
\item Curso de Akka Avanzado (provisto por \textit{Typesafe}).
\item Curso de desarrollador SCRUM (provisto po \textit{Kleer}).
\item Certificado de cumplimiento del curso en l\'inea \textit{Software Engineering for Software as a Service} dictado por profesores de la universidad de Berkeley en California. Abril de 2012 (\textit{Coursera}).
\item Certificado de cumplimiento del curso en l\'inea \textit{Machine Learning} dictado por un profesor de la universidad de Stanford. Julio de 2012 (\textit{Coursera}).
\item Certificado de cumplimiento del curso en l\'inea \textit{Programming Languages} dictado por un profesor de la universidad de Washington. Julio de 2013 (\textit{Coursera}).
\end{rSubsection}
\end{rSection}

%----------------------------------------------------------------------------------------
%   OSS SECTION
%----------------------------------------------------------------------------------------

\begin{rSection}{Contribuciones a Proyectos Open Source}

\begin{rSubsection}{}{}{}{}
\item \href{https://github.com/scalaz/scalaz/pull/750}{Scalaz}
\end{rSubsection}

\end{rSection}

%----------------------------------------------------------------------------------------

\begin{rSection}{Lenguajes}

\begin{tabular}{ @{} >{\bfseries}l @{\hspace{6ex}} l }
Ingl\'es: & Escrito - Nivel Medio. \\
			& Hablado - Nivel Medio. \\
			& Le\'ido - Nivel Alto. \\ 
\end{tabular}

\end{rSection}
%----------------------------------------------------------------------------------------
%	TECHNICAL STRENGTHS SECTION
%----------------------------------------------------------------------------------------

\begin{rSection}{Habilidades t\'ecnicas}

\begin{tabular}{ @{} >{\bfseries}l @{\hspace{6ex}} l }
Lenguajes de programaci\'on & \\ \textbf{(en orden de destreza):}& Scala, Java, Javascript, Python, Octave/Matlab. \\
Metodolog\'ias: & Scrum. Familiarizado con BDD y TDD. \\
Desarrollo web:& CSS b\'asico, Angular.js, Ember.js, jQuery, GWT.\\
%Desarrollo m\'ovil: & Android y Windows Phone\\
Plataformas empresariales: & Java EE 6. Spring.\\%, familiarizado con Ruby On Rails.\\
%Ofim\'atica: & Microsoft Word and Excel, \LaTeX \\
Bases de datos: & PostgreSQL, MySQL, Oracle. \\
Tecnolog\'ias NoSQL: & MongoDB, Hadoop. \\
Testing: & ScalaTest, Specs2, Mockito y conocimiento b\'asico de ScalaCheck. \\
%Toolkits: & Akka, Spray.\\
Sistemas de versionamiento: & Git, SVN. 
%Herramientas de programaci\'on: & Git, Eclipse. \\
\end{tabular}

\end{rSection}
%----------------------------------------------------------------------------------------
%	EXAMPLE SECTION
%----------------------------------------------------------------------------------------

%\begin{rSection}{Section Name}

%Section content\ldots

%\end{rSection}

%----------------------------------------------------------------------------------------

\end{document}

%%%%%%%%%%%%%%%%%%%%%%%%%%%%%%%%%%%%%%%%%
% Medium Length Professional CV
% LaTeX Template
%
% This template has been downloaded from:
% http://www.LaTeXTemplates.com
%
% Original author:
% Trey Hunner (http://www.treyhunner.com/)
%
% Important note:
% This template requires the resume.cls file to be in the same directory as the
% .tex file. The resume.cls file provides the resume style used for structuring the
% document.
%
%%%%%%%%%%%%%%%%%%%%%%%%%%%%%%%%%%%%%%%%%

%----------------------------------------------------------------------------------------
%   PACKAGES AND OTHER DOCUMENT CONFIGURATIONS
%----------------------------------------------------------------------------------------

\documentclass{resume} % Use the custom resume.cls style

\usepackage[left=0.4in,top=0.35in,right=0.4in,bottom=0.35in]{geometry} % Document margins
\usepackage{hyperref}
\name{Miguel Vil\'a} % Your name
\address{
%    (+571)~$\cdot$~540~$\cdot$~8404 \\ 
    \href{https://github.com/miguel-vila/}{miguel-vila @ github} \\
    \url{miguelvilag@gmail.com} \\
    Mobile phone: +44$\cdot$0744$\cdot$8026$\cdot$462
}
\def\uniandes{Los Andes University }
\def\dept{Systems Engineering Dept. }
\def\depart_uniandes{\dept y Computaci\'on, \uniandes}
\def\recurse_center{Recurse Center }

\hypersetup{
  colorlinks=true,
  urlcolor=blue,
}

\begin{document}

%\begin{rSection}{Perfil profesional}
%Me considero una persona responsable, honesta y trabajadora. Procuro balancear soluciones eficientes y solidas %con objetivos y necesidades concretas. Tengo habilidades para el trabajo en grupo y poseo iniciativa y curiosidad. Domino diversas tecnolog\'ias y soy capaz de aprender r\'apidamente otras nuevas.

%\end{rSection}

%\begin{rSection}{Intereses acad\'emicos}
%\begin{rSubsection}{}{}{}{}
%\item \textbf{Almacenamiento y procesamiento de grandes vol\'umenes de datos no estructurados:} Retos involucrados en el dise\~{n}o e implementaci\'on de sistemas que soporten el manejo de altas cantidades de informaci\'on no estructurada.
%\item \textbf{Procesamiento de lenguaje natural y miner\'ia de textos:} Aplicaciones de procesamiento de lenguaje natural que permitan generar valor a partir del an\'alisis de informaci\'on textual.
%\end{rSubsection}
%\end{rSection}
%----------------------------------------------------------------------------------------
%   WORK EXPERIENCE SECTION
%----------------------------------------------------------------------------------------

\begin{rSection}{Professional experience}

\begin{rSubsection}{Senior Software Engineer}{United Kingdom}{\href{https://www.disneystreaming.com/}{Disney Streaming Services}}{January 2017 - Present}
\begin{rList}
\item Working as part of multiple engineering teams building and maintaining Disney+ and ESPN+, Disney’s streaming platforms. Some of the technologies I use include Scala, Python, TypeScript and several AWS products.
\item Developing and maintaining services related to commerce operations including subscription lifecycle management and payment method storage. I collaborate with other teams in order to deliver cross-cutting features related to these services.
\begin{rSublist}
  \item Creating proposals for new components that improve the API and domain events of the subscription management system. Collaborating with other engineers in order to define the design and behavior of these new components. This will result in an improvement in the quality of the data used in analytics and better visibility over the subscriptions lifecycle.
  \item Developing and maintaining a component that detects when subscriptions issued by partners overlapped with existing subscriptions for the same user. This will result in the automatic pause of the subscription or a discount, depending on specific business rules. This provided a better experience for these users and costs savings in customer services operations.
\end{rSublist}
\item As part of a different team I developed services for user authentication, account management and profile management. This included several microservices written in Scala.
\begin{rSublist}
\item Executed performance tests against those services in order to define sensible scaling policies.
\item Helped to establish the infrastructure of multiple projects.
\end{rSublist}
\end{rList}
\end{rSubsection}

\begin{rSubsection}{Software Engineer}{Colombia}{\href{https://www.s4n.co/}{s4n}}{January 2013 - June 2016}
\begin{rList}
\item Technical leader of a small team of engineers. Responsible for technical decisions and training new team members in the use of the tech stack.
\item Developed enterprise software for a variety of companies, primarily insurance companies. Used a variety of technologies: Java, Scala and JavaScript amongst others.
\item Developed a REST API using Scala for an international client in the United States.
\end{rList}
\end{rSubsection}

%------------------------------------------------

\begin{rSubsection}{Teaching Assistant for Design and analysis of algorithms.}{Colombia}{\uniandes}{August 2011 - May 2012}
\begin{rList}
\item Evaluated student’s homework and exams. Organized and lectured classes in preparation for exams.
\end{rList}
\end{rSubsection}

%------------------------------------------------

\begin{rSubsection}{Software Engineer}{Colombia}{CIACUA, \uniandes}{March 2011 - August 2011}
\begin{rList}
\item Worked in the software reengineering process of an application used for the design and simulation of water distribution systems. Documented the design in UML and fixed bugs found in a version update.
\end{rList}
\end{rSubsection}

\end{rSection}
%----------------------------------------------------------------------------------------
%   EDUCATION SECTION
%----------------------------------------------------------------------------------------

\begin{rSection}{Education}

\begin{rSubsection}{\href{https://www.recurse.com/about}{Recurse Center}}{August 2016 - November 2016}{United States of America}
\item Self-directed, educational retreat for programmers to focus deeply on areas of programming outside their specialization. Specifically, I developed projects related to distributed systems, concurrency and functional programming.
\end{rSubsection}

\begin{rSubsection}{\uniandes}{January 2009 - December 2012} {Colombia}
\item BSc in Systems and Computing Engineering  
\end{rSubsection}

\end{rSection}

\begin{rSection}{Achievements and Awards}
\begin{rList}
%\item First place in the admission exams for Systems Engineering in the Colombia's National University (Bogot\'a, October 2008)
\item \href{http://losandesfoundation.org/cms31/index.php/quiero-estudiar}{\textsl{``Quiero estudiar"}} scolarship. This allowed me to study my undergrad at Los Andes University (October 2008).
\item Classified to the latin american regional International Collegiate Programming Contest, 2011.
\item Classified to the latin american regional International Collegiate Programming Contest, 2012.
\end{rList}
\end{rSection}

% \begin{rSection}{Complementary Education}
% \begin{rSubsection}{}{}{}{}
% \item SCRUM developer (provided by \textit{Kleer}).
% \item Advanced Akka (provided by \textit{Typesafe}).
% \item Fulfillment of the online course \textit{Software Engineering for Software as a Service} taught by professors from Berkeley University at California. April 2012 (\textit{Coursera}).
% \item Fulfillment of the online course \textit{Machine Learning} taught by a professor from Stanford University. July 2012 (\textit{Coursera}).
% \item Fulfillment of the online course \textit{Programming Languages} taught by a professor of Washington's University. July 2013 (\textit{Coursera}).
% \end{rSubsection}
% \end{rSection}

%----------------------------------------------------------------------------------------

% \begin{rSection}{Languages}

% \begin{tabular}{ @{} >{\bfseries}l @{\hspace{6ex}} l }
% Spanish: & Native Language.
% \end{tabular}

% \begin{tabular}{ @{} >{\bfseries}l @{\hspace{6ex}} l }
% English: & Advanced.
% \end{tabular}

% \end{rSection}
%----------------------------------------------------------------------------------------
%   TECHNICAL STRENGTHS SECTION
%----------------------------------------------------------------------------------------

\begin{rSection}{Technical Skills}

\begin{tabular}{ @{} >{\bfseries}l @{\hspace{6ex}} l }
Programming Languages & \\ \textbf{(in order of skill):}& Scala, Java, Python, Javascript. \\
Infrastructure: & Mostly AWS. Cloudformation, DynamoDB, Kinesis, ECS, CloudWatch. \\
% Versioning Systems: & Git, SVN. \\
% Basic knowledge of: & Python, Octave/Matlab. \\
%Languages I'm learning: & Haskell, Clojure, ClojureScript. \\
%Methodolog\'iess: & Scrum. Familiarized with BDD and TDD. \\
%Web Development:& Basic CSS, Angular.js, Basic Ember.js, jQuery, GWT.\\
%Desarrollo m\'ovil: & Android y Windows Phone\\
%Enterprise Frameworks: & Java EE 6. Spring.\\%, familiarizado con Ruby On Rails.\\
%Ofim\'atica: & Microsoft Word and Excel, \LaTeX \\
%Relational Databases: & PostgreSQL, MySQL, Oracle. \\
%NoSQL: & MongoDB, Hadoop. \\
%Testing: & ScalaTest, Specs2, Mockito and basic knowledge of ScalaCheck. \\
%Toolkits: & Akka, Spray.\\
%Herramientas de programaci\'on: & Git, Eclipse. \\
\end{tabular}

\end{rSection}

% ----------------------------------------------------------------------------------------
%   OSS SECTION
% ----------------------------------------------------------------------------------------

\begin{rSection}{Open Source Software Contributions}
\begin{rList}
\item \textbf{\href{https://github.com/scalaz/scalaz/pull/750}{scalaz \#750}}: Added a new functionality. Was able to do this after learning some functional programming concepts.
% \item \textbf{\href{https://github.com/dnvriend/akka-persistence-jdbc/pull/10}{akka-persistence-jdbc \#10}}: Fixed a bug based on a suggestion by other developer.
\item \textbf{\href{https://github.com/zio/zio-metrics/pull/53}{zio-metrics \#53}}: Improved the API of a metrics library, making it easier to use.
\end{rList}
\end{rSection}

%----------------------------------------------------------------------------------------
%   EXAMPLE SECTION
%----------------------------------------------------------------------------------------

%\begin{rSection}{Section Name}

%Section content\ldots

%\end{rSection}

%----------------------------------------------------------------------------------------

\end{document}
